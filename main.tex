% --- ここからコピー ---
\documentclass[12pt]{ltjsreport} % LuaLaTeX専用の頑丈なクラス

% 必要なパッケージ
\usepackage{graphicx}
\usepackage{url}
\usepackage{amsmath,amssymb}
\usepackage[hidelinks]{hyperref} % リンク機能

\begin{document}

% === 表紙 (Title Page) ===
\begin{titlepage}
    \centering
    \vspace*{10mm}
    
    {\Large 修士論文} \\
    \vspace{40mm}
    
   % タイトル
    {\LARGE \bfseries 家電情報を活用したユーザー状況取得に関する研究} \\
    
    \vspace{50mm}
    
    % 学籍番号・氏名
    {\Large
        2410040 \quad 岸 雄斗 \\
    }
    
    \vspace{10mm}
    
    % 指導教員
    {\large 主指導教員:丹 康雄 教授} \\
    
    \vspace{\fill}
    
    % 所属
    {\Large
        北陸先端科学技術大学院大学 \\
        先端科学技術研究科 \\
        (情報科学) \\
    }
    
    \vspace{10mm}
    
    % 提出年月
    {\Large 令和8年 2月}
    \vspace*{10mm}
    
\end{titlepage}
% === 表紙ここまで ===

% ページ番号の設定(ローマ数字)
\pagenumbering{roman}



% ======================
% Abstract (English)
% ======================
\newpage
\thispagestyle{plain}

\begin{center}
{\large Abstract}
\end{center}

\vspace{2mm}
In recent years, with the advancement of smart home technologies, various services utilizing data obtained from home appliances and facilities have attracted increasing attention. Applications such as energy management systems that optimize lighting and air conditioning according to residents’ situations, as well as monitoring services for elderly people living alone, are expected to improve both quality of life and household safety. To further enhance these services, it is essential to accurately estimate user status within a home, including occupancy, the number of residents, and their approximate locations.

However, many conventional monitoring and activity recognition systems rely on the installation of dedicated devices such as cameras or motion sensors. While these approaches can achieve high estimation accuracy, they raise concerns regarding privacy invasion and require additional installation costs and maintenance efforts, which hinder their widespread adoption in ordinary households.

To address these challenges, this research proposes a method for estimating user status by utilizing operation logs and usage data generated by existing smart home appliances, such as lighting systems, air conditioners, and air purifiers. The proposed approach integrates appliance operation data with physical spatial information, including floor plans and appliance placement, and applies machine learning techniques to estimate occupancy status, the number of residents, and their approximate locations in real time.

Furthermore, when appliance data alone is insufficient for accurate estimation, a minimal number of low-cost additional sensors, such as motion sensors, are selectively incorporated. The effectiveness of the proposed system is evaluated through experiments that compare different sensor configurations. In particular, the evaluation focuses on the reduction in the number of additional sensors required and the overall system cost, in comparison with conventional sensor-intensive approaches. The results of this study are expected to contribute to the development of practical, privacy-conscious, and cost-effective user status estimation methods, thereby promoting the deployment of smart home services in monitoring, safety, and disaster response applications.

% ======================
% 概要 (Japanese)
% ======================
\newpage
\thispagestyle{plain}

\begin{center}
{\large 概要}
\end{center}

\vspace{2mm}
近年、スマートホーム技術の進展に伴い、家庭内に設置された家電や設備から取得されるデータを活用したエネルギー管理や見守りサービスが注目されている。居住者の状況に応じて照明や空調を制御することで省エネルギーを実現する取り組みや、高齢者の在宅状況を把握する見守りサービスなど、生活の質と安全性を向上させる応用が期待されている。これらのサービスを高度化するためには、住宅内における居住者の在宅・不在、人数、さらには大まかな位置といったユーザー状況を正確に把握することが重要である。

一方で、既存の多くの見守り・行動認識システムでは、カメラや人感センサーなどの専用機器を新たに設置することを前提としている。このような手法は、高精度な推定が可能である反面、プライバシー侵害への懸念や導入コストの増大、設置・保守の手間といった課題を抱えており、一般家庭への広範な普及を妨げる要因となっている。

そこで本研究では、照明、空調機器、空気清浄機など、家庭内に既に導入されているスマート家電が生成する稼働履歴や操作情報に着目し、これらを主要な情報源として活用することで、居住者の在宅状況、人数、ならびに住宅内での大まかな位置を推定する手法を提案する。提案手法では、家電データと住宅の間取りや家電設置位置といった物理空間情報を統合し、機械学習手法を用いてユーザー状況をリアルタイムに推定するシステムを構築する。

また、家電データのみでは推定が困難な場合に限り、人感センサーなどの安価な専用センサーを最小限追加する構成を検討し、家電データ単独の場合との性能差を定量的に評価する。評価実験では、従来の多数の専用センサーを用いたシステムと比較し、ユーザー状況推定に必要な追加センサー数の削減効果および導入コストの観点から提案手法の有効性を検証する。本研究の成果は、プライバシーへの配慮と低コスト性を両立した実用的なユーザー状況取得技術の確立に寄与し、スマートホームサービスの社会実装や防災・見守り分野への応用に貢献することが期待される。

% ======================
% 目次
% ======================
\tableofcontents
\newpage

% 本文(算用数字)
\pagenumbering{arabic}

% ======================
% 第1章 はじめに
% ======================
\chapter{はじめに}

\section{研究背景}

\section{研究目的}

\section{研究の新規性と意義}

\section{本論文の構成}

% ======================
% 第2章 関連研究・関連技術
% ======================
\chapter{関連研究・関連技術}

\section{居住者状況推定に関する研究}

\subsection{専用センサーを用いた手法}

\subsection{スマートホームデータを用いた手法}

\section{スマートホームおよび家電データ}

\subsection{スマート家電の概要}

\subsection{家電から取得可能なデータ}

\section{ユーザー状況推定技術}

\subsection{在宅・不在推定}

\subsection{人数推定}

\subsection{位置推定}

\section{本研究の位置づけ}

% ======================
% 第3章 提案手法
% ======================
\chapter{提案手法}

\section{提案手法の概要}

\section{データ収集基盤}

\subsection{スマート家電データの取得}


\section{ユーザー状況推定手法}

\subsection{特徴量設計}

\subsection{機械学習モデル}

\subsection{追加センサーの役割}

\subsection{家電データとの統合}

% ======================
% 第4章 システム実装
% ======================
\chapter{システム実装}

\section{システム全体構成}

\section{データ処理・推定}

\section{ユーザーインターフェース}

\section{実装環境}

% ======================
% 第5章 評価
% ======================
\chapter{評価}

\section{評価実験の概要}

\section{評価指標}

\section{家電データのみを用いた推定結果}

\section{追加センサーを用いた推定結果}

\section{追加センサー削減効果の評価}


% ======================
% 第6章 考察
% ======================
\chapter{考察}

\section{推定精度に関する考察}

\section{センサー構成に関する考察}

\section{実環境への適用可能性}

\section{本研究の限界}

% ======================
% 第7章 おわりに
% ======================
\chapter{おわりに}

\section{結論}

\section{今後の課題}

% ======================
% 謝辞・参考文献
% ======================
\chapter*{謝辞}
\addcontentsline{toc}{chapter}{謝辞}

\chapter*{参考文献}
\addcontentsline{toc}{chapter}{参考文献}


\chapter*{謝辞}


\end{document}
% --- ここまでコピー ---