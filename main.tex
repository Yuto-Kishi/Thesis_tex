% --- ここからコピー ---
\documentclass[12pt]{ltjsreport} % LuaLaTeX専用の頑丈なクラス

% 必要なパッケージ
\usepackage{graphicx}
\usepackage{url}
\usepackage{amsmath,amssymb}
\usepackage[hidelinks]{hyperref} % リンク機能

\begin{document}

% === 表紙 (Title Page) ===
\begin{titlepage}
    \centering
    \vspace*{10mm}
    
    {\Large 修士論文} \\
    \vspace{40mm}
    
   % タイトル
    {\LARGE \bfseries 家電情報を活用したユーザー状況取得に関する研究} \\
    
    \vspace{50mm}
    
    % 学籍番号・氏名
    {\Large
        2410040 \quad 岸 雄斗 \\
    }
    
    \vspace{10mm}
    
    % 指導教員
    {\large 主指導教員:丹 康雄 教授} \\
    
    \vspace{\fill}
    
    % 所属
    {\Large
        北陸先端科学技術大学院大学 \\
        先端科学技術研究科 \\
        (情報科学) \\
    }
    
    \vspace{10mm}
    
    % 提出年月
    {\Large 令和8年 2月}
    \vspace*{10mm}
    
\end{titlepage}
% === 表紙ここまで ===

% ページ番号の設定(ローマ数字)
\pagenumbering{roman}



% ======================
% Abstract (English)
% ======================
\newpage
\thispagestyle{plain}

\begin{center}
{\large Abstract}
\end{center}

\vspace{2mm}
In recent years, with the advancement of smart home technologies, various services utilizing data obtained from home appliances and facilities have attracted increasing attention. Applications such as energy management systems that optimize lighting and air conditioning according to residents’ situations, as well as monitoring services for elderly people living alone, are expected to improve both quality of life and household safety. To further enhance these services, it is essential to accurately estimate user status within a home, including occupancy, the number of residents, and their approximate locations.

However, many conventional monitoring and activity recognition systems rely on the installation of dedicated devices such as cameras or motion sensors. While these approaches can achieve high estimation accuracy, they raise concerns regarding privacy invasion and require additional installation costs and maintenance efforts, which hinder their widespread adoption in ordinary households.

To address these challenges, this research proposes a method for estimating user status by utilizing operation logs and usage data generated by existing smart home appliances, such as lighting systems, air conditioners, and air purifiers. The proposed approach integrates appliance operation data with physical spatial information, including floor plans and appliance placement, and applies machine learning techniques to estimate occupancy status, the number of residents, and their approximate locations in real time.

Furthermore, when appliance data alone is insufficient for accurate estimation, a minimal number of low-cost additional sensors, such as motion sensors, are selectively incorporated. The effectiveness of the proposed system is evaluated through experiments that compare different sensor configurations. In particular, the evaluation focuses on the reduction in the number of additional sensors required and the overall system cost, in comparison with conventional sensor-intensive approaches. The results of this study are expected to contribute to the development of practical, privacy-conscious, and cost-effective user status estimation methods, thereby promoting the deployment of smart home services in monitoring, safety, and disaster response applications.

% ======================
% 概要 (Japanese)
% ======================
\newpage
\thispagestyle{plain}

\begin{center}
{\large 概要}
\end{center}

\vspace{2mm}
近年、IoT (Internet of Things) 技術やスマートホームの急速な普及に伴い、ネットワークに接続された家電製品や住宅設備から得られるデータの利活用が進んでいる。これらの機器から取得可能な稼働データは、単なる機器の遠隔操作や自動制御にとどまらず、家庭内のエネルギー消費の最適化(HEMS)、高齢者の見守り、あるいは防災・防犯といった、居住者の安全と快適性を支える高付加価値サービスの基盤として期待されている。このようなサービスを適切に提供するためには、居住者が「いつ」「どこで」「何人」滞在しているかといった、詳細なユーザー状況(コンテキスト)をリアルタイムかつ正確に把握することが不可欠である。

しかしながら、従来のユーザー状況推定システムの多くは、高精度な認識を実現するために、監視カメラや高密度なセンサーネットワーク、あるいはウェアラブルデバイスといった専用機器の新規導入を前提としている場合が多い。これらの手法は、技術的な推定精度においては優れているものの、カメラによる撮影が居住者に与える心理的な不快感やプライバシー侵害への懸念、さらには多数のセンサーを各部屋に設置・設定する際の物理的な作業負担や運用管理の手間が大きな障壁となり、一般家庭への広範な普及を妨げる要因となっている。

そこで本研究では、照明、空調機器、空気清浄機など、一般的な家庭に既に広く普及しつつあるスマート家電の操作情報や稼働履歴に着目し、これらを主要な情報源として活用することで、専用センサーへの依存度を極限まで低減したユーザー状況取得手法を提案する。提案手法の核心は、サイバー空間上のデータである「家電の操作ログ」と、フィジカル空間の情報である「住宅の間取り」や「家電の設置位置情報」を統合的に解析する点にある。これにより、居住者がどの家電を操作したかという断片的な情報から、現在の在宅・不在状況、滞在人数、および住宅内における大まかな位置情報を推定するアルゴリズムを構築する。

また、家電の操作頻度が低い時間帯や、家電が設置されていないエリアにおいては推定精度が低下する課題が想定される。これに対し、本研究では家電データのみに固執するのではなく、必要最低限の人感センサーなどを局所的に追加するハイブリッドな構成を検討し、プライバシーへの配慮と設置負担の軽減を両立させる。評価実験では、実際のスマートホーム環境においてデータを収集し、家電データのみを用いた場合と、少数の追加センサーを併用した場合の推定精度を比較検証する。これにより、従来の多数の専用センサーを要するシステムに対し、本提案手法がどれだけセンサー数を削減しつつ実用的な推定精度を維持できるか、その有効性と限界を明らかにする。
% ======================
% 目次
% ======================
\tableofcontents
\newpage

% 本文(算用数字)
\pagenumbering{arabic}

% ======================
% 第1章 はじめに
% ======================
\chapter{はじめに}

\section{研究背景}

\section{研究目的}

\section{研究の新規性と意義}

\section{本論文の構成}

% ======================
% 第2章 関連研究・関連技術
% ======================
\chapter{関連研究・関連技術}

\section{居住者状況推定に関する研究}

\subsection{専用センサーを用いた手法}

\subsection{スマートホームデータを用いた手法}

\section{スマートホームおよび家電データ}

\subsection{スマート家電の概要}

\subsection{家電から取得可能なデータ}

\section{ユーザー状況推定技術}

\subsection{在宅・不在推定}

\subsection{人数推定}

\subsection{位置推定}

\section{本研究の位置づけ}

% ======================
% 第3章 提案手法
% ======================
\chapter{提案手法}

\section{提案手法の概要}

\section{データ収集基盤}

\subsection{スマート家電データの取得}


\section{ユーザー状況推定手法}

\subsection{特徴量設計}

\subsection{機械学習モデル}

\subsection{追加センサーの役割}

\subsection{家電データとの統合}

% ======================
% 第4章 システム実装
% ======================
\chapter{システム実装}

\section{システム全体構成}

\section{データ処理・推定}

\section{ユーザーインターフェース}

\section{実装環境}

% ======================
% 第5章 評価
% ======================
\chapter{評価}

\section{評価実験の概要}

\section{評価指標}

\section{家電データのみを用いた推定結果}

\section{追加センサーを用いた推定結果}

\section{追加センサー削減効果の評価}


% ======================
% 第6章 考察
% ======================
\chapter{考察}

\section{推定精度に関する考察}

\section{センサー構成に関する考察}

\section{実環境への適用可能性}

\section{本研究の限界}

% ======================
% 第7章 おわりに
% ======================
\chapter{おわりに}

\section{結論}

\section{今後の課題}

% ======================
% 謝辞・参考文献
% ======================
\chapter*{謝辞}
\addcontentsline{toc}{chapter}{謝辞}

\chapter*{参考文献}
\addcontentsline{toc}{chapter}{参考文献}


\chapter*{謝辞}


\end{document}
% --- ここまでコピー ---