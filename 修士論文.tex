% --- ここからコピー ---
\documentclass[12pt]{ltjsreport} % LuaLaTeX専用の頑丈なクラス

% 必要なパッケージ
\usepackage{graphicx}
\usepackage{url}
\usepackage{amsmath,amssymb}
\usepackage[hidelinks]{hyperref} % リンク機能

\begin{document}

% === 表紙 (Title Page) ===
\begin{titlepage}
    \centering
    \vspace*{10mm}
    
    {\Large 修士論文} \\
    \vspace{40mm}
    
   % タイトル
    {\LARGE \bfseries 家電情報を活用したユーザー状況取得に関する研究} \\
    
    \vspace{50mm}
    
    % 学籍番号・氏名
    {\Large
        2410040 \quad 岸 雄斗 \\
    }
    
    \vspace{10mm}
    
    % 指導教員
    {\large 主指導教員:丹 康雄 教授} \\
    
    \vspace{\fill}
    
    % 所属
    {\Large
        北陸先端科学技術大学院大学 \\
        先端科学技術研究科 \\
        (情報科学) \\
    }
    
    \vspace{10mm}
    
    % 提出年月
    {\Large 令和8年 2月}
    \vspace*{10mm}
    
\end{titlepage}
% === 表紙ここまで ===

% ページ番号の設定(ローマ数字)
\pagenumbering{roman}



% ======================
% Abstract (English)
% ======================
\newpage
\thispagestyle{plain}

\begin{center}
{\large Abstract}
\end{center}

\vspace{2mm}
In recent years, advances in smart home technologies have enabled the collection and utilization of various types of data generated by home appliances and residential facilities. Such data have been increasingly used for applications including energy management, safety monitoring, and support services for elderly residents. In order to provide these services in a more adaptive and context-aware manner, it is essential to accurately estimate user status within a home environment, such as occupancy, the number of residents, and their approximate locations.

However, many existing systems for monitoring or user status estimation rely on the installation of dedicated devices, including cameras, motion sensors, or indoor positioning systems. While these approaches can achieve high accuracy, they often raise concerns regarding privacy invasion and require additional installation and maintenance costs, which hinder their practical deployment in ordinary households.

To address these challenges, this research proposes a user status estimation method that primarily utilizes operation logs and usage histories obtained from existing smart home appliances, such as lighting systems, air conditioners, and air purifiers. By focusing on appliance-level data that are already available in many households, the proposed approach aims to reduce dependence on privacy-sensitive and costly sensing devices. Furthermore, physical spatial information, including room layouts and appliance placement, is incorporated to enhance the interpretability and effectiveness of user status estimation.

The proposed method applies machine learning techniques to integrated appliance data in order to estimate occupancy status, the number of residents, and coarse-grained location information in real time. In situations where appliance data alone are insufficient, a minimal number of low-cost additional sensors are selectively introduced to complement the estimation process, thereby achieving a balance between estimation accuracy and system cost.

To evaluate the effectiveness of the proposed approach, comparative experiments are conducted under different sensor configurations. The evaluation focuses on the reduction in the number of additional sensors required and the associated deployment cost, while maintaining acceptable estimation performance. The results indicate that the proposed method enables practical and privacy-conscious user status estimation, contributing to the realization of low-cost smart home services for monitoring, safety, and energy-efficient living environments.

% ======================
% 概要 (Japanese)
% ======================
\newpage
\thispagestyle{plain}

\begin{center}
{\large 概要}
\end{center}

\vspace{2mm}
近年、スマートホーム技術の進展に伴い、家庭内に設置された家電や設備から取得されるデータを活用したエネルギー管理や見守りサービスが注目されている。居住者の状況に応じて照明や空調を制御することで省エネルギーを実現する取り組みや、高齢者の在宅状況を把握する見守りサービスなど、生活の質と安全性を向上させる応用が期待されている。これらのサービスを高度化するためには、住宅内における居住者の在宅・不在、人数、さらには大まかな位置といったユーザー状況を正確に把握することが重要である。

一方で、既存の多くの見守り・行動認識システムでは、カメラや人感センサーなどの専用機器を新たに設置することを前提としている。このような手法は高精度な推定が可能である反面、居住者のプライバシー侵害への懸念や、導入コストの増大、設置・保守にかかる手間といった課題を抱えており、一般家庭への広範な普及を妨げる要因となっている。

そこで本研究では、空調機器、空気清浄機など、家庭内に既に導入されているスマート家電が取得した内蔵データや操作情報に着目し、これらを主要な情報源として活用することで、居住者の在宅状況、人数、ならびに住宅内での大まかな位置を推定するシステムを構築した。本システムでは、スマート家電の動作データに加え、住宅の間取り情報や家電の設置位置といった物理空間情報を統合し、機械学習手法を用いることでユーザー状況をリアルタイムに推定することを可能とした。

また、家電データのみでは十分な推定精度が得られない場合に備え、人感センサーなどの安価な専用センサーを最小限追加する構成を検討し、家電データ単独の場合との推定性能の違いを定量的に評価した。評価実験では、従来の多数の専用センサーを用いたシステムと比較し、ユーザー状況推定に必要な追加センサー数の削減効果および導入コストの観点から本システムの有効性を検証した。その結果、提案システムはプライバシーへの配慮と低コスト性を両立しつつ、実用的なユーザー状況取得が可能であることを示した。

本研究の成果は、既存のスマート家電を活用した低負担なユーザー状況取得技術として、スマートホームサービスの社会実装を促進するものであり、平常時の見守りや省エネルギー制御のみならず、災害時における居住者状況把握といった応用への展開も期待される。

% ======================
% 目次
% ======================
\tableofcontents
\newpage

% 本文(算用数字)
\pagenumbering{arabic}

% ======================
% 第1章 はじめに
% ======================
\chapter{はじめに}

本章では、本研究の背景、目的、本文の構成について述べる。

\section{研究背景}

近年,IoT 技術や通信基盤の発展により,家庭内に設置された家電や設備がネットワークに接続され,それらから多様なデータを取得・活用できるスマートホーム環境が普及しつつある.スマートホームでは,家電の稼働状態や操作履歴といった情報を活用することで,エネルギー消費の最適化や,居住者の生活を支援するサービスの実現が期待されている.特に,居住者の在宅状況や住宅内での位置といったユーザー状況を把握することは,状況に応じた家電制御や見守りサービスを実現する上で重要な要素である.

このようなユーザー状況を取得するために,これまで多くの研究やシステムでは,カメラや人感センサーなどの専用センサーを住宅内に設置する手法が採用されてきた.これらの手法は高精度な推定が可能である一方,居住者のプライバシー侵害への懸念や,導入・保守にかかるコスト,設置作業の負担といった課題を抱えている.そのため,一般家庭において継続的に利用されるシステムとしては,必ずしも導入しやすいとは言えない状況にある.

一方で,近年のスマート家電は,稼働状態や内部センサー値などの情報を家電自身が保持・出力する機能を備えており,これらのデータは既存の住宅環境において追加の機器を設置することなく取得可能である.このような家電由来のデータを活用することで,専用センサーへの依存を低減しつつ,居住者の状況を推定できる可能性がある.しかし,家電データのみを用いたユーザー状況取得手法については,推定可能な情報の範囲や実用性について十分な検討がなされているとは言い難い.



\section{研究目的}

本研究の目的は,家庭内に既に導入されているスマート家電が生成する稼働履歴や操作情報を主要な情報源として活用し,居住者の在宅状況,人数,および住宅内での大まかな位置を取得するシステムを構築し,その有効性を検証することである.

具体的には,照明,空調機器,空気清浄機などのスマート家電から取得可能なデータと,住宅の間取り情報や家電の設置位置といった物理空間情報を統合し,機械学習手法を用いてユーザー状況を推定する.さらに,家電データのみでは推定が困難な場合に限り,人感センサーなどの安価な専用センサーを最小限追加する構成を検討し,推定性能への影響を評価する.

本研究では,従来の多数の専用センサーを用いた手法と比較し,ユーザー状況推定に必要なセンサー数の削減効果および導入コストの観点から,提案システムの有効性を明らかにすることを目指す.これにより,プライバシーへの配慮と低コスト性を両立した,実用的なユーザー状況取得手法の確立を目的とする.


\section{本論文の構成}

本論文の構成を以下に示す.第1章では,本研究の背景および目的について述べ,本研究の位置づけを明確にする.第2章では,居住者状況推定に関する関連研究およびスマートホーム技術について整理し,本研究の特徴を示す.第3章では,スマート家電データを活用したユーザー状況取得手法を提案し,その概要と推定手法について説明する.第4章では,構築したシステムの全体構成および実装内容について述べる.第5章では,評価実験の方法と結果を示し,提案システムの有効性を検証する.第6章では,得られた結果に基づく考察を行い,実環境への適用可能性や課題について述べる.最後に第7章では,本研究の結論と今後の課題についてまとめる.


% ======================
% 第2章 関連研究・関連技術
% ======================
\chapter{関連研究・関連技術}

\section{居住者状況推定に関する研究}

\subsection{専用センサーを用いた手法}

\subsection{スマートホームデータを用いた手法}

\section{スマートホームおよび家電データ}

\subsection{スマート家電の概要}

\subsection{家電から取得可能なデータ}

\section{ユーザー状況推定技術}

\subsection{在宅・不在推定}

\subsection{人数推定}

\subsection{位置推定}

\section{本研究の位置づけ}

% ======================
% 第3章 提案手法
% ======================
\chapter{提案手法}

\section{提案手法の概要}

\section{データ収集基盤}

\subsection{スマート家電データの取得}


\section{ユーザー状況推定手法}

\subsection{特徴量設計}

\subsection{機械学習モデル}

\subsection{追加センサーの役割}

\subsection{家電データとの統合}

% ======================
% 第4章 システム実装
% ======================
\chapter{システム実装}

\section{システム全体構成}

\section{データ処理・推定}

\section{ユーザーインターフェース}

\section{実装環境}

% ======================
% 第5章 評価
% ======================
\chapter{評価}

\section{評価実験の概要}

\section{評価指標}

\section{家電データのみを用いた推定結果}

\section{追加センサーを用いた推定結果}

\section{追加センサー削減効果の評価}


% ======================
% 第6章 考察
% ======================
\chapter{考察}

\section{推定精度に関する考察}

\section{センサー構成に関する考察}

\section{実環境への適用可能性}

\section{本研究の限界}

% ======================
% 第7章 おわりに
% ======================
\chapter{おわりに}

\section{結論}

\section{今後の課題}

% ======================
% 謝辞・参考文献
% ======================
\chapter*{謝辞}
\addcontentsline{toc}{chapter}{謝辞}

\chapter*{参考文献}
\addcontentsline{toc}{chapter}{参考文献}


\chapter*{謝辞}


\end{document}
% --- ここまでコピー ---